\pdfoutput=1
\documentclass{article}
\usepackage[body={6.7in,9.2in},top=.8in,left=.9in,nohead]{geometry}

\pagestyle{empty}

\newcommand{\myhead}[1]{\vspace{6pt}\noindent\textbf{#1}}

\begin{document}

\begin{center}
 \textbf{Xspread 2.4 Quick Start} 
\end{center}

Executive summary of \textsf{xspread 2.4}, a re-working of the
venerable X11 spreadsheet program for the Agenda VR3 Linux-based PDA.
This is not a substitute for the full manual, which you should read if
you wish to make the best use of \textsf{xspread}.  There's also
built-in help documentation, which is accessed by typing \texttt{?}.

\myhead{Menus}

Hit \texttt{/} (slash) to get into the menu system.  You can scroll
the menus with the Left/Right buttons and select items with the
stylus.  The \texttt{/} command can also be used to back up the menu
tree if need be.

\myhead{Getting data into xspread}

\textsf{Xspread} reads its own format (ASCII) worksheets.  To get
pre-existing data into this format, use the utility \textsf{xsdata},
supplied as part of the \textsf{xspread} distribution.

To enter data manually, just start \textsf{xspread} and type in the
data values.  In the default entry mode, typing a digit, \texttt{+},
\texttt{-} or \texttt{=} starts a numeric entry (either a constant or
a numerical formula).  String entries are started by typing
\texttt{"} (for a centered label), \texttt{<} (left flush label) or
\texttt{>} (right flush label).

\myhead{Formulas}

These work as in most spreadsheets.  You can reference cell contents
by putting the address of the cell (e.g.\ \texttt{B20}) into a
formula.  The \texttt{\$} symbol is used for absolute references.  To
enter a cell's address into the command line, tap on the cell with the
stylus.  (There are other ways of doing this---see the manual.)
There's a reasonably wide selection of built-in functions, whose names
all start with \texttt{@} (see the manual or built-in help).

\myhead{Navigating}

The Agenda's Left, Right, Up and Down buttons all move the cell
highlight by single rows or columns.  You can move the highlight onto
a given cell by tapping on the cell.  There are various cursor-motion
keystrokes too (see the manual).  When you're editing a formula, the
Left and Right buttons move the text cursor within the command line,
but you can get the button-motion focus back onto the worksheet by
tapping on the bar that displays the column headings. (And you can get
the focus back onto the command line by tapping on it.)

\myhead{Editing}

To get a cell open for editing, select it (i.e.\ get the highlight
onto it) then tap on the top line.  An editable version of the cell's
contents will appear.  At the edit prompt the keystrokes
\texttt{Ctrl-A}, \texttt{Ctrl-E} and \texttt{Ctrl-K} do their Emacs
things: start of line, end of line, and delete to end of line.

\myhead{Ranges}

Many commands operate on ranges of cells.  The range separator is the
colon, as in \texttt{A2:B10}.  When a range is required as part of a
command, \textsf{xspread} will (usually) insert a trailing colon if
you tap a cell to enter its address into the command line.  Some
commands accept ``degenerate'' ranges (single cell addresses): if you
don't want the automatically supplied colon, then backspace over it.

\myhead{Canceling commands}

\texttt{Esc} or \texttt{Ctrl-G} gets you out of most things.

\myhead{Getting out}

Besides the \texttt{Quit} menu item, a typed \texttt{Q} (not
case-sensitive) quits \textsf{xspread}, with a query regarding saving
the data if anything has changed since the last save.

\myhead{Command hints}

Where a command requires specific arguments, the nature of the
required input is shown in square brackets on the second line of the
display.  Where a command wants a quoted string, \textsf{xspread}
supplies the opening quote; the closing quote is generally not needed,
unless a further argument follows the quoted string.

\myhead{Graphs}

I had to fudge things to get graphs working at all on the Agenda;
please don't expect too much!  Only the XY and line graph types are
available, and text elements (titles, legends) are likely to disappear
or mess up the graph.  

\vspace{8pt}

\begin{raggedleft}
Allin Cottrell\\
\texttt{cottrell@wfu.edu}\\
May, 2001\\
\end{raggedleft}










\end{document}
