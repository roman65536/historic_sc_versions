\documentclass[titlepage]{article}
\usepackage[body={6.5in,9in},top=1in,left=1in,nohead]{geometry}
\usepackage[pdftex]{hyperref}

\setcounter{secnumdepth}{2}
\setcounter{tocdepth}{2}

\title{Xspread Reference Manual}
\author{James Cornelius\\ Michael Frey\\ Dan Gruber\\ Fang Wang\\
        \\
        Manual Updated by Robert Parbs II\\
      Further updated by Allin Cottrell for version 2.4}
\date{May, 2001}
%

\newcommand{\titem}[1]{\item[{\tt #1}]}

%namelist generates a list with an item width of
% your choice; form: \begin{namelist}{widestitem}
\newcommand{\namelistlabel}[1]{\mbox{#1}\hfil}
\newenvironment{namelist}[1]{%
\begin{list}{}
  {
   \let\makelabel\namelistlabel
   \settowidth{\labelwidth}{#1}
   \setlength{\leftmargin}{1.1\labelwidth}
  }
}{%
\end{list}}

\newcommand{\twiddle}{\texttt{\~{}}}  % tilde character, verbatim
\newcommand{\ctrl}[1]{\texttt{\^{}#1}}

\begin{document}

\maketitle 

\tableofcontents

\vspace{24pt}

\begin{center}
\textbf{A note on version 2.4}
\end{center}

Version 2.4 of \textsf{xspread} represents a ``revival'' of the
program, designed primarily for use on the Agenda VR3 Linux-based
Personal Digital Assistant (PDA).  This manual is based on the version
included with \textsf{xspread 2.3}, which is part of the Slackware
Linux distribution.  I have, however, made many changes: the changes
fall into the following categories:

\begin{enumerate}
\item Editorial amendments in the interest of greater concision.
\item Removal of ``wishful'' references to features that were in the
  manual, but not actually implemented in version 2.3 of the program.
\item Removal of references to features that have been eliminated in
  version 2.4, in the interest of a smaller \textsf{xspread} binary.
\item Addition of references to new features added in version 2.4,
  e.g.\ the Linear Regression option and the functions
  \texttt{@entropy}, \texttt{@log2}, \texttt{@norm}, \texttt{@normcdf}
  and \texttt{@rand}.
\end{enumerate}

Website for version 2.4: \verb+http://www.ecn.wfu.edu/~cottrell/agenda/+

\begin{raggedleft}
Allin Cottrell\\
Wake Forest University\\
May 2001\\
\end{raggedleft}

%\clearpage

\section{Introduction}

\textsf{Xspread} is a spreadsheet program which runs under the X
Window system.  It supports many standard spreadsheet features, such
as:
\begin{itemize}
\item Cell entry and editing
\item Worksheet size: 702 columns by unlimited rows
\item File reading and writing
\item Absolute and relative cell references
\item Numeric and string (``label'') data in cells
\item Left or right justification for labels
\item Row and column insertion and deletion
\item Hiding and unhiding of rows and columns
\item Range names
\item Manual or automatic recalculation
\item Numeric, relational and Boolean operators
\item Function references, including references to external programs
\item Use of the mouse in pointing and menu selection
\end{itemize}

The structure and operation of the spreadsheet is similar to but not
identical with popular spreadsheets such as \textsf{Lotus 1-2-3} and
its clones.  Like other spreadsheets, the workspace is arranged into
rows and columns of cells.  Each cell can contain a number, a label
(i.e.\ character string), or a formula which evaluates to a number or
label.

You can start the program with or without specifying a file to be read
in.  This file must be a worksheet in \textsf{xspread} format.  Such
files are created either by saving a worksheet after entering data
manually, or with the auxiliary program \textsf{xsdata}, a filter
which reads plain space-separated data and writes it out in the
appropriate format.  If a file is specified on the command line,
\textsf{xspread} attempts to locate and read in the file.  If it is
successful, \textsf{xspread} starts with the file's contents in the
workspace.  If it is unsuccessful or no file is specified on the
command line, \textsf{xspread} starts with the workspace empty.

For a tutorial introduction, type:

\begin{center}
  \texttt{xspread examples/tutorial.xsw}
\end{center}

This directory also contains other spreadsheet templates which you may
be interested in.

To start \textsf{xspread}, type the program name followed by any
command flags you want to use and then by the optional file name.  The
full form of the command line is:

\begin{center}
  {\tt xspread [-c] [-h] [-m] [-n] [-C] [-R] [-fn font]
    [filename]}
\end{center}

The optional flags given above have the following meanings:

\begin{itemize}
  
  \titem{-c} Recalculation is done in column order.  When
    \textsf{xspread} recalculates, it will start at the top of the
    leftmost column, and recalculate the all the cells from top to
    bottom.  Then, it will recalculate the next column in the same
    order.  It will continue in this fashion until it has recalculated
    the  rightmost column.

    \textsf{Xspread} does NOT support natural order recalculation.

    Default: Row order recalculation.
  
  \titem{-h} Display command line help.
  
  \titem{-m} Start with manual recalculation.  With this option, the
    spreadsheet will recalculate values only when the \texttt{@}
    command is used.  With automatic recalculation, the spreadsheet
    recalculates values whenever a cell's
    contents change.\\
    Default: Automatic recalculation.
  
  \titem{-n} Use ``alternative'' data entry mode.  See
    section~\ref{cellentry} below for details.

  \titem{-fn} Change the font size to whatever, e.g.\ \texttt{-fn 9x15}.
  
  \titem{-C} The action after the {\tt Enter} key is released is to
  move the cursor down the current column.
  
  \titem{-R} The action after the {\tt Enter} key is released is
  to move the cursor right in the current row.\\
  Default: Stay at the current position.
\end{itemize}

\section{Using the Worksheet}

\subsection{Worksheet Structure}

The \textsf{xspread} window is divided into four regions.
The top line is used for displaying the cell address the cursor is on,
displaying cell values, and entering commands.  The second region
consists of the second and the third line. Here \textsf{xspread}
displays messages or options for the \texttt{/} (menu) command. In the
latter case the third line is used to show a short description of the
highlighted option.  A third region, immediately under the third line
and along the left edge of the window, shows the column addresses and
row addresses.  The fourth region is the worksheet work space.

The worksheet has 702 columns labeled alphabetically \texttt{A}
through \texttt{ZZ} (\texttt{A} through \texttt{Z} and \texttt{AA}
through \texttt{ZZ}).  The number of rows only depends on the
available memory. Rows are numbered from \texttt{0} on.

Where a row meets a column, the intersection is called a cell.  Cells
have addresses which consist of their column letter(s) and row number.
Examples of cell addresses are \texttt{A1}, \texttt{E56}, and
\texttt{AH187}.  The upper left corner has cell address \texttt{A0}.
The cell address occupied by the cursor is indicated on the top line.

If a cell's numeric value is wider than the column width, the cell is
filled with asterisks.  If the cell's label string is wider than the
column width, the display of the label is truncated at the start of
the next non-blank cell in the same row.

The \textsf{xspread} window has two cursors.  The cell cursor
highlights the current cell.  The character cursor shows up when you
type a command on the top line.

Commands are fed to \textsf{xspread} using the mouse (or stylus, on
the Agenda) and via various keystrokes.

This manual indicates control key combinations by showing a caret
(\ctrl{}) immediately prior to the control key's letter.  For example,
\texttt{Ctrl-A} is shown as \ctrl{A}.

\subsection{Navigating the Worksheet}

\subsubsection{Moving the Cursor One Cell at a Time}

The arrow keys (or the directional buttons on the Agenda) may be used
to move the cell cursor by single rows or columns, if the ``focus'' is
on the worksheet.  The other place the focus can be is the top line,
where the left and right keys (or buttons) can be used to move the
text cursor when you're editing a command.  To see where the focus is,
look at the top left corner of the region displaying the row and
column headings: when the focus is on the worksheet a reverse-video
asterisk \texttt{*} is displayed there; if the focus is on the top
line this asterisk disappears.  To shift the focus onto the worksheet,
click on the bar showing the column headings; to shift it onto the top
line, click there.

In addition, you can move the cursor around the worksheet via various
control key sequences.  (The control key commands always are available
even if the character cursor is on the top line.)

\begin{itemize}
\item[\ctrl{B}] (back) and \ctrl{F} (forward) move the
  cursor left and right, respectively.
\item[\ctrl{P}] (previous) and \ctrl{N} (next) move the
  cursor up and down, respectively.
\end{itemize}
  
Additional \textsf{vi}-like cursor control commands are available if
the character cursor is \textit{not} on the top line of the window:

\begin{itemize}
\titem{h} (back) and \texttt{l} (forward) move the cursor left
  and right.
\titem{k} (up) and \texttt{j} (down) move the cursor up and
  down.
\item[\ctrl{H}] and the spacebar move the cursor back and
  forward, respectively.
\end{itemize}

\subsubsection*{Larger Cursor Moves}

\begin{itemize}
  
  \titem{\^{}} Go to the top row of the worksheet.
  
  \titem{\#} Go to the bottom row.
  
  \titem{0} Left edge: Move the cursor to column A.
  
  \titem{\$} Right edge: Move the cursor to the last column of the
  worksheet.
  
  \titem{b} Scans the cursor backwards (i.e. to the left and up) to
    the previous valid (non-blank) cell.
  
  \titem{w} Scans the cursor forwards (i.e. to the right and down) to
    the next valid (non-blank) cell.
  
  \titem{\ctrl{E}d} Goes to the next non-blank cell in the indicated
    direction.  The character \texttt{d} must be replaced by one of
    the valid cursor direction indicators (i.e., \ctrl{B}, \ctrl{F},
    \ctrl{P}, or \ctrl{N}).  When you execute this command, if the
    cursor is on a blank cell, it goes in the indicated direction
    until it reaches the first non-blank cell.

\end{itemize}

\subsubsection{Moving to Specific Locations}

\texttt{g} goes to a specific cell.  You are prompted for a cell
address, range name, a string expression surrounded by quotes, or a
number.  If you specify a cell address or a range name,
\textsf{xspread} goes directly to that cell, or the starting (upper
left) cell of the range.  If you supply a quoted string,
\textsf{xspread} will search for a cell containing the given
expression.  If you specify a number, \textsf{xspread} will search for
a cell containing that number.

\textsf{Xspread}'s searches proceed from the current cell to the end
of the worksheet, then wrap to cell \texttt{A0} and continue from
there forward to the current cell.

\subsubsection{Moving to Specific Locations via the mouse}

The mouse can also be used for navigation: clicking on a cell with the
left or middle mouse button will move the highlight onto that cell.
The third mouse button is used to invoke the menu system.  (Agenda: tap a
cell with the stylus to move the cell cursor onto that cell.)

\subsection{Cell Entry and Editing}
\label{cellentry}

Cells can contain either numeric or string constants or expressions.

\subsubsection{Data entry}

There are two data entry modes: ``quick numeric'' mode (the default)
and ``alternative'' mode.

The two modes have this in common: to enter a \textbf{label or
  string}, you must first enter one of these characters: \texttt{"},
\verb+>+, or \verb+<+.  (Entering a letter first does NOT start a
label, since bare letters are used as \textsf{xspread} commands.)

\begin{itemize}
\titem{"} indicates that the label will be centered in the
  current cell.
\titem{<} indicates that the label will placed flush left.
\titem{>} indicates that the label will be flush right.
\end{itemize}

The two entry modes differ in regard to entry of \textbf{numerical}
constants or expressions.  In quick numeric entry mode, a numerical
entry may be started with \texttt{=}, a digit (\texttt{0-9}),
\texttt{+} or \texttt{-}.  In alternative mode, a numeric entry must
start with the \texttt{=} sign.  In this case, unprefixed digits,
\texttt{+} and \texttt{-} have a different function, as explained in
the following subsection.

In all cases, \textsf{xspread} prompts you for the expression on the
top line.

\subsubsection{Cell Editing Commands}

When the highlight is on a given cell in the worksheet, and its
contents are displayed on the top line, you can ``open'' the cell for
editing by clicking on the top line: the line displaying the current
content will change to a prompt allowing you to edit the content.
This prompt somewhat resembles the \textsf{bash} command line.  You
can use the arrow keys to move left or right, and the keystrokes
\ctrl{A}, \ctrl{E} and \ctrl{K} can be used to go to the beginning of
the line, go to the end of the line, and delete to the end of the
line, respectively.

To commit changes made at this prompt, press \texttt{Enter}; to
abandon any changes, press \texttt{Esc} or \ctrl{G}.

In addition various letter keys perform editing commands, provided the
cell is \textit{not} already ``opened'' as described above.

\begin{itemize}

  \titem{e} edit the numeric value associated with the current cell.
    \textsf{Xspread} will display the current numeric expression on
    the top line with the character cursor at the end of the numeric
    expression.  

  \titem{E} edit the label that already exists in the current cell.
    \textsf{Xspread} will display the current label on the top line
    with the character cursor at the end of the label.  
  
    \titem{c} copy the last marked cell to the current cell.  For
    compatibility with other programs, \ctrl{V} (``Paste'') is a
    synonym.
  
    \titem{m} mark a cell for later use by the \texttt{c} command.
    For compatibility with other programs, \ctrl{C} (``Copy'') is
    a synonym.
  
  \titem{x} clear (erase) the current cell.  You can use any of the
    pull commands to retrieve cell contents that were previously
    deleted.
  
    \titem{+} in ``alternative'' entry mode, add the value of its
    argument (i.e. a number typed before the \texttt{+}) to the value
    of the current cell and store the result in the current cell.
    If no numeric argument is given the increment is 1.0.
  
    \titem{-} in ``alternative'' entry mode, subtract the value of its
    argument (i.e. a number typed before the \texttt{-}) from the value
    of the current cell and store the result in the current cell.
    If no numeric argument is given the decrement is 1.0.

\end{itemize}

To adjust the alignment of a string label in a given cell, select the
cell and hit one of \texttt{"} (center), \texttt{<} (left) or
\texttt{>} (right).  The original cell content will be shown on the
top line with the new alignment.  Just hit \texttt{Enter} if you don't
want to change the string itself.

\subsection{Formulas, Cell Expressions, and Functions}

\subsubsection{Formulas}

Without formulas, a computer spreadsheet would not be any better
than its paper counterpart.  It is the ability to enter and
recalculate formulas that gives an electronic spreadsheet its real
power.  Formulas can link result cells to other cells in the
spreadsheet.  These other cells can, in turn, reference still other
cells so that a recalculation of the entire spreadsheet can have a
cascade effect.  Through formulas, a single cell can affect cells
throughout the entire worksheet.

Formulas can reference cells either through the cell's address (e.g.
\texttt{K20}) or through defined range names.  Both cell addresses and
range names can be either relative, absolute, or a combination of the
two.  Relative cell addresses and range names change when the cell's
formula is copied to another position in the worksheet.  Absolute cell
addresses and range names do not change when the cell's formula is
copied to another position in the worksheet.

\subsubsection{Cell References}

The method of specifying absolute cell addresses follows the
convention of \textsf{Lotus 1-2-3}: absolute references are preceded
by a dollar sign, \texttt{\$}.  The dollar sign can precede either the
column reference, row reference, or both.  Here are some examples,
each of which references cell \texttt{K20} in a different way:

\begin{namelist}{\$K\$20xx}
\titem{K20} Both the column and row
    references change when the cell is copied.
  
\titem{\$K\$20} Both the column reference
    and row reference remain fixed when the cell is copied.
  
\titem{\$K20} The column reference remains
    fixed but the row reference changes when the cell is copied.
  
\titem{K\$20} The column reference changes
    but the row reference remains fixed when the cell is copied.
\end{namelist}

These conventions also hold on defined (named) ranges---see
section~\ref{subsec:range} below.  In general, range references vary
when formulas containing them are copied, but if the range is defined
with \texttt{\$} references, these do not change.

\subsubsection{Operators}

The numeric operators for formulas are as follows:

\vspace{6pt}
\begin{tabular}{ll}
 \texttt{+} & Addition\\
 \texttt{-} & Subtraction\\
 \texttt{*} & Multiplication\\
 \texttt{/} & Division\\
 \texttt{\^{}} & Exponentiation (raise to a power)\\
 \texttt{\%} & Modulus or remainder\\
 \texttt{()} & Parentheses can be used to change the order of operations
\end{tabular}
\vspace{6pt}

You can use relational operators to compare two numeric expressions to
see if they satisfy the specified relation.  The result is a logical
value, either true (1) or false (0).  The relational operators are:

\vspace{6pt}
\begin{tabular}{ll}
\texttt{=} & Equal to\\
\texttt{!=} & Not equal to\\
\texttt{>} & Greater than\\
\texttt{>=} & Greater than or equal to\\
\texttt{<} & Less than\\
\texttt{<=} & Less than or equal to
\end{tabular}
\vspace{6pt}

Logical operators may be used to construct compound logical expressions.
The logical operators are:
\label{logical_ops}

\vspace{6pt}
\begin{tabular}{ll}
\twiddle & Logical NOT\\
\texttt{\&} & Logical AND\\
\texttt{|} & Logical OR\\
\end{tabular}
\vspace{6pt}

You can use the conditional operator to test for a condition and take
action depending on whether that condition is true or false (i.e.,
perform an IF test).  The conditional operator is:

\begin{itemize}
  \item[] \texttt{e1?e2:e3} If expression \texttt{e1} is
  true, return the value of expression \texttt{e2}, otherwise return
  the value of expression \texttt{e3}.
\end{itemize}

\subsubsection{Function References}

\textsf{Xspread} supports a number of functions that make it easy to
perform calculations of a specific nature.  Formulas can reference any
of the functions defined in the Function Reference
(section~\ref{sec:funref}).  You can use these function references
just as you would any cell or range reference.  All functions begin
with the \texttt{@} character.  Since the \texttt{@} character by
itself is used as a command character (to recalculate the
spreadsheet), you must prefix \texttt{@} in a function reference with
a \texttt{+} sign or \texttt{-} sign if a function reference is the
first item in a formula.

\subsection{Toggle Commands}
\label{sec:toggle}

\textsf{Xspread} has several optional settings which operate as toggle
switches.  Most switches have just two settings: the toggle commands
change the setting of the selected switch to its opposite without
your having to go through the \texttt{/} menu tree.  (See also
section~\ref{menuoptions} below.)

All of these commands are of the form \texttt{Ctrl-Tx}
(\texttt{\^{}Tx}), where \texttt{x} is replaced by the letter denoting
the option that you want to toggle.  The settings of all toggle
options are saved with the worksheet when it is written to file.  The
toggle options and their code letters are as follows:

\begin{itemize}
\titem{a} {\bf Automatic / Manual Recalculation.}  If automatic
    recalculation is set, every change to the spreadsheet will cause
    the spreadsheet to be recalculated.  If manual recalculation is
    set, \textsf{xspread} does not recalculate the spreadsheet unless
    you explicitly issue the recalculation command, \texttt{@}.

\titem{c} {\bf Recalculate by Columns / Rows.} The default is to
    recalculate by rows.
    
\titem{e} {\bf External Function Execution.}  If external
    function execution is enabled, \textsf{xspread} calls such
    functions whenever the screen is updated.  Otherwise any such
    functions are not called during screen updates.  If external
    functions are referenced in the worksheet and they are disabled,
    \textsf{xspread} prints a warning each time the screen is
    updated.  See also the entry for \texttt{@ext} in
    section~\ref{sec:funref} below.

\titem{g} {\bf Grid Lines Show/Hide.}  Show (the default) or hide
    worksheet grid lines, separating the cells.
    
\titem{i} {\bf Round to Infinity.} If set, .5 is always rounded
    up, as opposed to the default which is to round .5 to the nearest
    even number (known as ``banker's rounding'').
    
\titem{n} {\bf Quick Numeric Entry.} If set, you can start numeric
    entry with any digit, a plus sign, or a minus sign.  If not set,
    you must start a numeric entry with \texttt{=} (and you can use
    \texttt{+} and \texttt{-} to increment or decrement cell entries).
    Quick numeric mode is the default.

\titem{\$} {\bf Dollar Prescale.}  If set, numeric amounts are
    automatically scaled by .01 when you enter them into cells.  This
    allows users to avoid typing the decimal points in monetary
    amounts.  If not set (the default), numeric amounts are not scaled.
    
\titem{r} {\bf Newline Action.} Three-way toggle for the cursor
    motion on hitting \texttt{Enter} at the conclusion of editing a
    cell entry (none, move down one row, or move right one column).
    The default is no motion.
    
\titem{z} {\bf Set Newline Action Limit.}  Set the limit to the
    ``newline action'' (see above) to the current cell.  If the
    newline action is to move down, setting (e.g.) cell \texttt{A5} as
    the limit will cause the cursor motion to wrap to a new column
    after row 5, i.e.\ the cursor motion sequence will be \texttt{A4},
    \texttt{A5}, \texttt{B0}, \texttt{B1} and so on.  This can be
    useful for filling in a block of data of known size.

\end{itemize}

\subsection{Miscellaneous Commands}

\begin{itemize}
\titem{Q} \texttt{q} Exit from \textsf{xspread}

\item[\ctrl{G}], \texttt{Esc} Abort the current command.
  
\titem{?} Bring up an index to on-line help.  The index will display
  a list of topics together with the letter that allows you to select
  a particular topic.  The help facility is not context sensitive.
  
  \titem{Tab} The ``point'' command.  When composing an expression on
  the top line, you can use \texttt{Tab} to define a range by
  pointing, instead of by typing cell addresses.  First use the motion
  keys to get the highlight onto the cell that starts the desired
  range, then press \texttt{Tab}.  Use the motion keys again to move
  to the end of the desired range (the intervening range will be
  highlighted as you go).  Hit \texttt{Tab} a second time to enter the
  selected range into the expression on the top line.

  \item[\ctrl{L}] Redraw the screen.
    
  \item[\ctrl{R}] Redraw the screen, highlighting cells containing
    constant numeric values.  This may be useful for showing values
    which you need to provide or update.
    
  \item[\ctrl{X}] Redraw the screen, highlighting cells containing
    expressions.  \textsf{Xspread} shows all expressions as formulas,
    not their current values.  All expressions are displayed as
    left-justified text.  This command makes it easier to check
    expressions.

\end{itemize}

Two \texttt{Ctrl} commands insert current cell information into the
top line, if an expression is being composed there:
 
\begin{itemize}
  \item[\ctrl{V}] {\bf Numeric Value.} Inserts the numeric value of the
  current cell.
  
  \item[\ctrl{W}] {\bf Cell Expression.} Inserts the expression
  attached to the current cell, if any.  If there is no expression,
  this command inserts \texttt{?}.
\end{itemize}

In addition, while an expression is being composed or edited, clicking
on a cell with the mouse (or tapping on it with the stylus on the
Agenda) will insert that cell's address into the top line, at the
cursor position.

%%%%%%%%%%%%%%%%%%%%%%%%%%%%%%%%%%%%%%%%%%%%%%%%%%%%%%%%%%%%%%%
\section{Alphabetical Command Reference}
\label{sec:cmdref}

%\newcommand{\usage}{\vspace{1ex}\noindent \textit{Usage:}}
\let\usage\relax

\subsection[Column and Row Commands]{/C \ \  Column/Row}
        
The \texttt{Column/Row} commands perform various operations on entire
columns or rows in the worksheet (deletion, insertion and so on).
Details on the individual commands follow.


\subsubsection{/CA \ \  Column/Row Append}

Inserts a new row or column immediately following the cursor
position and copies the contents of the current row or column into the
newly inserted row or column.

\usage{}
\begin{enumerate}\itemsep -2pt
\item Move the cursor to the cell in a row or column where you want
  the new row or column to be inserted.  Rows are inserted below and
  columns to the right.
\item Type \texttt{/CA}
\item Type \texttt{R} for Row or \texttt{C} for Column.
\item The row or column is inserted and filled with the copied values.
\end{enumerate}

\subsubsection{/CD \ \ Column/Row Delete}
        
Deletes a row or column from the worksheet.  The remaining rows or
columns are renumbered to close the space.  

\usage{}
\begin{enumerate}\itemsep -2pt
\item Move the cursor to the row or column you want to delete.
\item Type \texttt{/CD}
\item Type \texttt{R} for row or \texttt{C} for column.
\item Press \texttt{Enter}.  The row or column at the current cursor
  position is deleted.
\end{enumerate}


\subsubsection{/CF \ \ Column/Row Format}
        
Sets column width and the numeric display format for a column.  (There
is no command to format a row.)

\usage{}
\begin{enumerate}\itemsep -2pt
\item Move the cursor to the column you want to format.
\item Type \texttt{/CF}
\item Enter the column width and the number of digits to follow the
  decimal point.  E.g.\ if you specify \texttt{8 1} the column will be
  8 characters wide and show one decimal place.  Values are rounded
  off to the least significant digit displayed.
\end{enumerate}

\subsubsection{/CH \ \  Column/Row Hide}

Hides the current row or column: this keeps it from being displayed
but it remains in the worksheet.

\usage{}
\begin{enumerate}\itemsep -2pt
\item Move the cursor to the row or column you want to format.
\item Type \texttt{/CH}
\item Type \texttt{R} for row or \texttt{C} for column.
\item Press \texttt{Enter}.  The row or column at the current cursor
  position is hidden.
\end{enumerate}

\subsubsection{/CI \ \ Column/Row Insert}

Inserts a row or column into the worksheet at the current cursor
position.  A new row appears immediately below the cursor and a new
column immediately to the right.

\usage{}
\begin{enumerate}\itemsep -2pt
\item Move the cursor to a cell in the row or column where you want
  the new row or column to be inserted.
\item Type \texttt{/CI}
\item Type \texttt{R} for Row or \texttt{C} for Column.
\end{enumerate}


\subsubsection{/CP \ \     Column/Row Pull}

Reinserts (``pulls'') deleted information back into the worksheet at
the current cursor location.  \texttt{/CPR} inserts enough rows to
hold the last deleted set of cells.  \texttt{/CPC} inserts enough
columns to hold the last deleted set of cells.  \texttt{/CPM} (Merge)
does not insert rows or columns; it overwrites the cells beginning at
the current cursor location.

\usage{}
\begin{enumerate}\itemsep -2pt
\item Move the cursor to the position where you want the previously deleted
  information to appear.
\item Type \texttt{/CP}
\item Type \texttt{R} for row, \texttt{C} for column or \texttt{M} for
  merge.
\item Press \texttt{Enter}.  \textsf{Xspread} inserts the deleted
  information in the manner specified.
\end{enumerate}

\subsubsection{/CS \ \     Column/Row Show}

Shows (unhides) hidden rows or columns.  It is the reverse of the
\texttt{/CH} (\texttt{Column/Row Hide}) command.

\usage{}
\begin{enumerate}\itemsep -2pt
\item Type \texttt{/CS}
\item Type \texttt{R} for row or \texttt{C} for column.
\item Enter a range of rows or columns to be revealed.  The default
  action is the first range of rows or columns currently hidden.
\end{enumerate}

\subsubsection{/CV \ \     Column/Row Values}

Converts formulas in the affected rows or columns, inserting the
values which are in the cells when the command is executed.  This
``freezes'' the values.  

\usage{}
\begin{enumerate}\itemsep -2pt
\item Move the cursor to a cell in the row or column you want to
  convert from formulas to values.
\item Type \texttt{/CV}
\item Type \texttt{R} for row or \texttt{C} for column.
\end{enumerate}

\subsection[File Commands]{/F  \ \     File}
        
The \texttt{File} commands transfer information between the current worksheet
and files on disk.  As of version 2.4.4, \textsf{xspread} uses the
FLTK file browser to select file names.

%The \texttt{/FS} (Save) and \texttt{/FE} (Export)
%commands can pipe their output to a program.  In order to use this
%feature, type \texttt{| progname} to the prompt asking for a filename.
Details on the individual commands follow.

        
\subsubsection{/FO \ \     File Open}

Retrieves a worksheet file from disk.  Only files in
\textsf{xspread}'s native format can be opened in this way.  The
auxiliary program \textsf{xsdata} can be used to convert plain text data
files into \textsf{xspread} format.  See \texttt{man 1
  xsdata}.\footnote{This program was formerly known as \textsf{psc}.}

\subsubsection{/FM \ \     File Merge}

Merges the chosen file with the current worksheet.  To the degree
the data ranges in the specified file overlap the ranges of the
current worksheet, the existing data are overwritten.

\subsubsection{/FS \ \     File Save}

Saves the current worksheet or a specified range (see \texttt{/FR}
below) to disk in \textsf{xspread}'s native format.

\subsubsection{/FE \ \     File Export}

Exports the worksheet data or a specified range thereof (see below)
in a format suitable for use with other programs.  A submenu is
presented with a choice of export formats: \texttt{Screen} (plain
text, formatted as shown in the \textsf{xspread} worksheet window),
\texttt{CSV} (Comma Separated Values, suitable for opening in other
spreadsheet programs), \texttt{LaTeX} (\LaTeX{} tabular environment),
and \texttt{Tbl} (in the format of \textsf{tbl}, the table processor
for \textsf{troff}).  You can access these options via their initial
letters, \texttt{S}, \texttt{C}, \texttt{L} or \texttt{T}.

\subsubsection{/FR \ \     File Range}

Allows you to set a range of cells for use with the \texttt{Save} and
\texttt{Export} file commands.  If this option is not set, the range
defaults to the entire worksheet.
        
\subsection[Graph Commands]{/G  \ \     Graph}

The Graph commands set up and generate graphs using worksheet
data.  Submenus let you choose the \texttt{Type} of the graph (XY,
line or bar) and the data ranges to use (\texttt{X} and \texttt{A}
through \texttt{D}).  The \texttt{Options} submenu to \texttt{Graph}
offers further submenus for controlling details of the plot such as
titles, legends and grids.

\subsubsection{/G A-D\ \  Graph A-D (Data Ranges)}

These items let you designate up to four data ranges for plotting on
the Y axis.

\usage{}
\begin{enumerate}\itemsep -2pt
\item Type a letter from \texttt{A} through \texttt{D} at the Graph menu.
\item Respond to the prompt with a range such as \texttt{K1:K20}.
\item To specify more ranges, use this command again with a different
    letter for the next range.
\end{enumerate}

\subsubsection{/GE \ \  Graph Export}

This item allows you to export an \textsf{xspread} graph in the format
of \textsf{gnuplot}.  (Once your graph is in \textsf{gnuplot} you can
convert it to a wide range of other formats.)  You are prompted for
the name of the file to save, by default with the extension \texttt{.gp}.

\subsubsection{/GO \ \     Graph Options}

These commands allow you to add enhancements to your graph.  If you
save the worksheet, the graph options most recently selected are saved
with the data.

\begin{description}
\item[{\bf Legend }]{Adds text identifying each Y axis data range.}
\item[{\bf Format }]{Defines how graph information will be presented.}
\item[{\bf Titles }]{Adds titles at the top of the graph and along the
    X and Y axes.}
\item[{\bf Grid}]{ Adds horizontal and/or vertical grid lines.}
\item[{\bf Scale }]{Sets the upper and lower limits for the graph axes.}

\end{description}
        
\subsubsection{/GOF\ \     Graph Options Format}

Controls the representation of data points in a line or XY graph. 
The default setting is \texttt{Symbols}: data points are represented by
marker symbols.  The \texttt{Lines} option uses lines to connect the
points; \texttt{Both} uses symbols as well as connecting lines.
        
\usage{}
\begin{enumerate}\itemsep -2pt
\item Type \texttt{F} at the Graph Options menu.
\item Select a single data range (\texttt{A} through \texttt{D}) or
  \texttt{Graph} for the entire graph.
\item Choose a format for the specified range: {\tt Lines}, {\tt
    Symbols} or {\tt Both}.
\item Continue choosing ranges or formats as desired.
\item To exit the Graph Format menu, press the \texttt{Esc} key.
\end{enumerate}

\subsubsection{/GOG\ \     Graph Options Grid}

Adds or clears grid lines from graphs.  Options are 
\texttt{Horizontal} grid lines, \texttt{Vertical} grid lines,
\texttt{Both} (a full grid) and \texttt{Clear} (no grid, only tick
marks).  \texttt{Clear} is the default.


\subsubsection{/GOL\ \     Graph Options Legend}

Defines strings to be used as identifying ``legends'' for each Y axis
data range.

\usage{}
\begin{enumerate}\itemsep -2pt
\item Type \texttt{L} from the Graph Options menu.
\item Specify the data range (\texttt{A} to \texttt{D}) to be
    identified with a legend.
\item Type the legend (up to 39 characters) and press \texttt{Enter}.
\item To
    exit the Graph Options Legend menu, press the \texttt{Esc} key.
    To create more than one legend, select another menu
    item.
\end{enumerate}

\subsubsection{/GOS\ \     Graph Options Scale}

These commands allow you to specify maxima and minima for the ranges
of the data on the X and/or Y axes.  The default is \texttt{Automatic}
(the graph is adjusted to include all points in each data range). 

\subsubsection{/GOT\ \     Graph Options Titles}

This command lets you define titles for the X and Y axes and for the
top of the graph.
\begin{description}
\item[{\bf First }]{Places a centered title at the top of the graph.}
\item[{\bf Second }]{Places a centered title under the first title
    line.}
\item[{\bf X }]{Places a label below the horizontal (X) axis.}
\item[{\bf Y }]{Puts a label beside the vertical (Y) axis.}
\end{description}

\usage{}
\begin{enumerate}\itemsep -2pt
\item Type \texttt{T} from the Graph Options menu.
\item Select an option for the position of the title.
\item At the prompt, type your title (up to 39 characters) and 
  press \texttt{Enter}.
\item To exit the Graph Options Titles menu, press the \texttt{Esc}
  key.  To create more than one title, select another item from
  the menu.
\end{enumerate}
            
\subsubsection{/GR \ \     Graph Reset}

This command resets all graph parameters to their default values
(graph type is \texttt{XY}, symbols are used for plotting, no grid,
axis scales are automatic, all data ranges are undefined, and any
titles and legends are erased).
        
\subsubsection{/GT \ \     Graph Type}

This menu lets you select the type of graph to be created, from the choices
\texttt{XY} (scatter diagram, the default), \texttt{Line} (suitable
for time series) and \texttt{Bar} (bar chart).  

\subsubsection{/GV \ \     Graph View}

\textsf{Xspread} displays the graph in a separate window.  Press any
key or mouse button (with the mouse pointer in the graph window) to
return to the \texttt{Graph} menu.

\subsubsection{/GX\ \      Graph X}

For XY graphs, sets the X data range.  For line and bar graphs, sets
the range of cell labels for the horizontal axis.  In the case of a
line graph, you can leave the X range undefined, in which case the
graphing routine will supply automatic labels for the horizontal axis
(consecutive integers representing the observation number).

\usage{}
\begin{enumerate}\itemsep -2pt
\item Type \texttt{X} at the Graph menu.
\item \textsf{Xspread} will give you the prompt
  {\tt range for X data:} or {\tt range for X labels:}
  Respond with a range such as \texttt{B1:B20} and press
  \texttt{Enter}.
\end{enumerate} 

\subsection[Linear Regression]{/L \ \      Linreg}

This item allows you to estimate a simple linear regression.  You are
prompted for a range of cells representing the dependent variable, 
then for a second range containing values of the independent variable,
and finally for a cell marking the top left corner of the results.
Note that the results occupy an area of 3 rows by 2 columns; it is
probably best to specify a cell in a blank column to the right of your
data to hold the results.

The printed results comprise the estimated value of the constant or
intercept of the regression, the estimated regression slope, the
estimated standard error of the slope coefficient---labeled
\texttt{s(slope)}---and the coefficient of determination, $R^2$.

\subsection[Matrix Manipulation]{/M \ \      Matrix}

Each command under the Matrix menu performs a particular matrix
function.

\begin{description}
  
\item[{\bf /MT Transpose}]Transpose a matrix: You are prompted for the
  range defining the source matrix, and then for the top left cell of
  the destination (result) range.
  
\item[{\bf /MA Addition}]Add two matrices: You are prompted for the
  range of the first matrix, then for the range of a second matrix,
  then for the top left cell of the destination range.  The matrices
  to be added must be of the same size.
  
\item[{\bf /MS Subtraction}]Subtract one matrix from another: You are
  prompted first for ``matrix 1'' which is the left-hand matrix, then
  for ``matrix 2'', the right-hand one, and then for the top left cell
  of the destination range.  Matrix 2 is subtracted from matrix 1 (if
  they are of the same dimensions).
  
\item[{\bf /MM Multiplication}]Multiply one matrix into another: You
  are prompted first for the left-hand matrix range, then for the
  right-hand range, then for the top left cell of the destination
  range.  If the two matrices are conformable for multiplication, the
  result is the second matrix pre-multiplied by the first.
  
\item[{\bf /MI Inversion}]Invert a Matrix: You are prompted for the
  range of the source matrix, which must be square, and then for the
  top left cell of the result range.  If the source matrix is
  non-singular the result is the inverse of the source.

\end{description}

\subsection[Setting Program Options]{/O \ \ Options}
\label{menuoptions}

These commands set various worksheet options.  (See
section~\ref{sec:toggle} above for an alternative means of accessing
these switches, and for greater detail on the options).  The
\texttt{Option} items are as follows.

\begin{description}

  \titem{Auto} Toggle the recalculation mode between automatic (the
  default) and manual (recalculation is done only in response to the
  \texttt{@} command).  
  
  \titem{Numeric} Toggle the cell entry mode between ``quick numeric''
  (the default) and ``alternative''.
  
  \titem{PreScale} Toggle the automatic prescaling of numeric entries
  by .01.
  
  \titem{Ext} Toggle the evaluation of ``external functions'' on
  recalculation of the worksheet.
  
  \titem{NL-Action} Three-way toggle for the cursor motion on hitting
  \texttt{Enter} at the conclusion of editing a cell entry (none, move
  down one row, or move right one column).
    
  \titem{Col/Row-Limits} Set the current cell as the limit for the
  ``newline action'' (see above).  
  
  \titem{Recalc} Toggle the recalculation order: by rows (the default)
  or by columns.

  \titem{Up} Toggle rounding of .5 between always up and to the nearest
  even (the default).  

  \titem{Iterations} Set the maximum number of
  iterations on recalculation (you are prompted for an integer).

  \titem{Grid lines} Toggle between showing and hiding grid lines that
   separate the cells of the worksheet.

\end{description}

\subsection[Quitting the Program]{/Q \ \       Quit}

The \texttt{Quit} command ends the current worksheet session.  If
changes have been made since the last time the worksheet was saved,
\textsf{xspread} will ask whether the most recent changes
should be saved prior to exiting: respond with \texttt{Y} to save the
changes or \texttt{N} to exit without saving the worksheet.

\subsection[Range Commands]{/R \ \      Range}
\label{subsec:range}
        
The Range commands affect a single cell or rectangular group of
adjacent cells.  Options are \texttt{Copy}, \texttt{Define} and
\texttt{Remove}, \texttt{Erase}, \texttt{Fill}, \texttt{Lock} and
\texttt{Unlock}, and \texttt{Values} (freeze the output from
formulas).  Details follow.
        
\subsubsection{/RC\ \      Range Copy}

Copies the values and formulas in the source range into the
destination range.  Relative cell references are adjusted for the new
position; absolute cell references are left alone.

CAUTION: The Range Copy command overwrites the contents of the
destination cells.

\usage{}
\begin{enumerate}\itemsep -2pt
\item Type \texttt{/RC}
\item At the \texttt{copy} prompt, supply the source range and the
  destination range.
\end{enumerate}
        
\subsubsection{/RD\ \      Range Define}

Lets you specify a name for a cell or range of cells.  Later, you
can use this name instead of cell references in formulas.  Range names
are case sensitive.  For example, ``\texttt{foo}'' and
``\texttt{Foo}'' are distinct range names.

\usage{}
\begin{enumerate}\itemsep -2pt
\item Type \texttt{/RD}
\item At the \texttt{define} prompt, enter a string enclosed in double
  quotes, followed by a range specification such as \texttt{A4:B8}.
\end{enumerate}
        
\subsubsection{/RE\ \      Range Erase}

Erases the contents of cells in a specified range.

\usage{}
\begin{enumerate}\itemsep -2pt
\item Type \texttt{/RE}
\item Specify a range at the \texttt{erase} prompt.  \textsf{Xspread}
  erases the contents of all cells in the range.
\end{enumerate}

\subsubsection{/RF \ \     Range Fill}

Fills a specified range of cells with a designated value.  All of the
cells can have the same value or each succeeding cell can differ from
the previous one by a stated increment.

\usage{}
\begin{enumerate}\itemsep -2pt
\item Type \texttt{/RF}
\item At the \texttt{fill} prompt, specify in order the range to be
  filled, the value to use in the first cell, and the amount by which
  the next cell should differ from the previous cell.  If all cells
  are to have the same value, the increment should be zero (0).
\end{enumerate}

\subsubsection{/RL \ \     Range Lock}

Protects a specified range against overwriting.  To remove the
protection, use \texttt{/RU}, \texttt{Range Unlock}.  
        
\subsubsection{/RS\ \      Range Show}

Shows the user all of the named ranges.

\usage{}
\begin{enumerate}\itemsep -2pt
\item Type \texttt{/RS}
\item \textsf{Xspread} displays in a separate window a list of the
  currently defined range names and the ranges that are denoted by
  those names.  Press any key or click the mouse on the list window
  to dismiss it.
\end{enumerate}
        
\subsubsection{/RR\ \      Range Remove}

Removes (undefines) a range name; it does not delete the cells in the
range.

\usage{}
\begin{enumerate}\itemsep -2pt
\item Type \texttt{/RU}
\item Specify the range to be ``removed''.  \textsf{Xspread} deletes
  the associated range name.
\end{enumerate}

\subsubsection{/RV\ \      Range Values}

Converts the results of formulas to the values they yield at
the time that the command is executed.

\usage{}
\begin{enumerate}\itemsep -2pt
\item Type \texttt{/RV}
\item At the prompt, specify the range.  \textsf{Xspread} then
  substitutes the values of the formulas for the formulas.  (There will
  be no apparent change in the screen display.)
\end{enumerate}

%%%%%%%%%%%%%%%%%%%%%%%%%%%%%%%%%%%%%%%%%%%%%%%%%%%%%%%%%%%%%%%%%%%%%
\section{Function Reference}
\label{sec:funref}

Functions can be used by themselves or as part of formulas in
\textsf{xspread}.  Their names begin with an \texttt{@} symbol.  Any
necessary arguments are enclosed in parentheses, immediately following
the function name.  The general form of a function reference is:
  \texttt{@function(}\textit{arg1}, \textit{arg2}, \ldots,
  \textit{argN})

\subsection{Argument types}

\begin{center}
\begin{tabular}{lp{5in}}
\texttt{date} & unix date: the number of
    seconds since midnight on January 1, 1970.\\
\texttt{format} & string containing a valid C-language
    format specification for converting character expressions to
    numeric and vice versa.\\
\texttt{i} & The interest rate per period on a loan or investment,
    expressed as a decimal fraction.\\
\texttt{n} & Any integer.\\
\texttt{pmt} & The payment made at the end of each term of a loan
    or investment.\\
\texttt{position} & Integer value specifying the position inside
    a character string.\\
\texttt{pv } & The present value of a series of payments.  The
    original amount of a loan or investment.\\
\texttt{range} & A range name or cell address.\\
\texttt{crange} & A range with an optional
    logical expression attached (see section~\ref{condrange} below).\\
\texttt{term} & The number of payment periods over the life of a
    loan or investment.\\
\texttt{string} & A character string enclosed in quotes or the cell
    address of a label.\\
\texttt{x,y} & Double precision floating point numbers or cells
    containing such numbers.\\
\end{tabular}
\end{center}

\subsection{Functions by type}

\begin{tabular*}{.65\textwidth}{@{\extracolsep{\fill}}llllll}

\multicolumn{5}{l}{Date and Time Functions}\\[4pt]
& {\tt @date} & {\tt @day} & {\tt @hour} & {\tt @minute}\\
& {\tt @month } & {\tt @now} & {\tt @second} & {\tt @year}\\[8pt]


\multicolumn{5}{l}{Financial functions}\\[4pt]
& {\tt @fv} & {\tt @pmt} & {\tt @pv} & {\tt @irr}\\[8pt]

\multicolumn{5}{l}{Lookup Functions}\\[4pt]
&   {\tt @index} & {\tt @lookup} & {\tt @stindex} \\[8pt]

\multicolumn{5}{l}{Mathematical Functions}\\[4pt]
&   {\tt @ceil} & {\tt @exp} & {\tt @fabs} & {\tt @floor} & {\tt @hypot}\\
&   {\tt @ln} & {\tt @log} & {\tt @log2} & {\tt @max} & {\tt @min}\\
&   {\tt @nval} & {\tt @pi} & {\tt @pow} & {\tt @rnd} & {\tt @sqrt}\\[8pt]

\multicolumn{5}{l}{Special Functions}\\[4pt]
&  {\tt @ext} & {\tt @norm} & {\tt @rand}\\[8pt]

\multicolumn{5}{l}{Statistical Functions}\\[4pt]
&   {\tt @avg} & {\tt @count} & {\tt @entropy} & {\tt @max} & {\tt @min}\\
&   {\tt @normcdf} & {\tt @prod} & {\tt @stddev} & {\tt @sum}\\[8pt]

\multicolumn{5}{l}{String Functions}\\[4pt]
&   {\tt @eqs} & {\tt @fmt} & {\tt @ston} & {\tt @substr} & {\tt @sval}\\[8pt]

\multicolumn{5}{l}{Trigonometric Functions}\\[4pt]
&   {\tt @acos} & {\tt @asin} & {\tt @atan} & {\tt @atan2} & {\tt @cos}\\
&   {\tt @dtr} & {\tt @rtd} & {\tt @sin} & {\tt @tan}\\

\end{tabular*}

\subsection{Alphabetical Function Reference}

\newcommand{\fnstart}[1]{%
  \vspace{1ex}\noindent\texttt{#1} }

\fnstart{@acos(x)} Returns the arc cosine, i.e.\ the
angle in radians whose cosine is $x$.  The argument $x$ must be in the
range $-1$ to 1.  The angle is in the range 0 to $\pi$.

\fnstart{@asin(x)} Returns the arc sine, i.e.\ the angle
in radians whose since is $x$.  The argument $x$ must be in the range
$-1$ to 1.  The angle is in the range $-\pi/2$ to $\pi/2$.

\fnstart{@atan(x)} Returns the arc tangent, i.e.\ the
angle in radians whose tangent is $x$.  The angle is in the range
$-\pi/2$ to $\pi/2$.

\fnstart{@atan2(x,y)} Returns the arc tangent, i.e.\ the
angle in radians whose tangent is $y/x$.  The angle is in the range
$-\pi$ to $\pi$.  This function distinguishes between angles that lie
in the first and third quadrants and those that are in the second and
fourth.

\fnstart{@avg(crange)} Returns the average (arithmetic mean) of the
values in the given (conditional) range.  If the range contains blank
cells, they are ignored.

\fnstart{@ceil(x)} Returns the smallest integer which is not less than
$x$.

\fnstart{@cos(x)} Returns the cosine of $x$.  The argument $x$ must be
in radians.

\fnstart{@count(crange)} Returns the number of non-empty cells in
 \texttt{crange}.
 
 \fnstart{@date(date)} Converts the unix date to a character string of
 the form \texttt{Nnn Mmm dd hh:mm:ss yyyy}, where

\begin{itemize}\itemsep -2pt
    \item[] \texttt{Nnn} is the name of the day of the week
    \item[] \texttt{Mmm} is name of the month of the year
    \item[] \texttt{dd} is the day of the month
    \item[] \texttt{hh:mm:ss} is the 24-hour time in hours, 
      minutes, and seconds
    \item[] \texttt{yyyy} is the year
\end{itemize}

\fnstart{@day(date)} Returns the day of the month given a unix date.

\fnstart{@dtr(x)} Converts the angle measurement $x$ in degrees to
radians.

\fnstart{@entropy(range)} Returns $H = -\sum p_i \log_2 p_i$, the
Boltzmann entropy or Shannon information of a set of probabilities,
$p_i$, contained in the given \texttt{range}.  The values in
\texttt{range} must satisfy the rules for probabilities: $0 \le p_i
\le 1$ for all $i$ and $\sum p_i = 1$.

\fnstart{@eqs(string1,string2)} Compares the values of two string
expressions: returns 1 if \texttt{string1} has the same value as
\texttt{string2}, otherwise returns 0.

\fnstart{@exp(x)} Returns the value of $e$ (2.718281828...) raised to
the power of $x$.  \texttt{@exp} is the inverse function of
\texttt{@ln}.

\fnstart{@ext(string,x)} This function allows the user to call
external functions from inside a spreadsheet.  The external function
must be a valid program that the shell can run when the function is
called.

\texttt{string} contains the program command line that is passed to the 
command interpreter.  $x$ is a numeric value which is passed to the
function named in string.  The value of $x$ is converted to character
format and concatenated to the end of string before the command
interpreter is called.

The result of \texttt{@ext} is a string containing the first line
which the external program prints to standard output.  Any additional
output to standard output, or any output to standard error, will mess
up the screen.  \texttt{@ext} returns a null string if external
function evaluation is disabled, \texttt{string} is null, or the
attempt to run the command fails.

\fnstart{@fabs(x)} Returns the absolute value of the number specified
by the argument.  The absolute value is either zero or the positive
value of the number.

\fnstart{@floor(x)} Returns the largest integer which is less than or
equal to the value of the argument.

\fnstart{@fmt(format,x)} Converts the numeric argument $x$ to a string
under the guidance of \texttt{format}, which must be a string
containing a C-language format specification on one of the following
patterns: \texttt{\%ew.d}, \texttt{\%Ew.d}, \texttt{\%fw.d},
\texttt{\%gw.d}, or \texttt{\%Gw.d}, where \texttt{w} gives the total
width of the field in characters and \texttt{d} gives the number of
characters to the right of the decimal point.

\fnstart{@fv(pmt,i,term)} Returns the future value of an ordinary
annuity with the payment made at the end of each term, at a fixed
interest rate.  The arguments are the periodic payment amount,
\texttt{pmt}, the interest rate per period, \texttt{i}, and the number
of periods, \texttt{term}.  The time interval for expressing the
interest rate and \texttt{term} must be the same.  For example, if
\texttt{term} is in months, the interest rate must be given on a
monthly basis.

\fnstart{@hour(date)} Returns the hour from a unix date.  The hours
are the number of hours since midnight.  Thus, 0 represents midnight
and 23 represents 11 p.m.

\fnstart{@hypot(x,y)} Returns the length of the hypotenuse of a right
triangle, i.e.\ $\sqrt{x^2 + y^2}$.

\fnstart{@index(n,range)} Returns the numeric contents of a cell
specified by the index number \texttt{n} and \texttt{range}, which is
any single row or column in the worksheet.  The range cells are
numbered from 1 to $n$, starting with the leftmost cell in the row or
the topmost cell in the column.

\fnstart{@irr(range)} Returns the Internal Rate of Return (i.e.\ the
discount rate which yields a net present value of zero) for the
specified range of cells, where the first cell is taken as the
negative of the initial investment and the subsequent cells hold
values representing the net income (positive) or outlay (negative) in
succeeding periods.  
 
\fnstart{@ln(x)} 

\fnstart{@log(x)} 

\fnstart{@log2(x)} These functions return the logarithm (to base $e$,
base 10, and base 2, respectively) of $x > 0$.

\fnstart{@lookup(x,range)}

\fnstart{@lookup(string,range)} These functions return the contents of
a cell from a ``table'', which can be either two rows or two columns.

The numeric version compares the value of $x$ to the table given by
the row or column \texttt{range}.  The function searches the row or
column for the last value less than or equal to $x$.  If
\texttt{range} is a row, the function returns the value in the next
row and the same column.  If \texttt{range} is a column, the function
returns the value in the same row and the next column.

The string version compares the value of \texttt{string} to the table
located in the row or column range.  The function searches the row or
column for an exact string match.  If \texttt{range} is a row, the
function returns the value in the next row and the same column.  If
\texttt{range} is a column, the function returns the value in the same
row and the next column.

\fnstart{@max(crange)}

\fnstart{@max(x1,x2,...)}  These functions return the largest value
specified by the arguments.  The arguments can be either a single
range or a list of numeric expressions separated by commas.

\fnstart{@min(crange)}

\fnstart{@min(x1,x2,...)}  These functions return the smallest value
specified by the arguments.  The arguments can be either a single
range or a list of numeric expressions separated by commas.

\fnstart{@minute(date)} Returns the number of minutes (0 to 59)
since the last whole hour, given a unix date.

\fnstart{@month(date)} Returns the number of the month (1 to 12),
given a unix date.

\fnstart{@norm} (No argument and no parentheses required.)  Returns a
pseudo-random number from the standard normal or Gaussian distribution
(with a mean of 0 and standard deviation of 1).

\fnstart{@normcdf(x)}  The normal cumulative density function: returns
the proportion of the standard normal distribution lying between
$-\infty$ and $x$.

\fnstart{@now} Returns the current unix date.

\fnstart{@nval(string,n)} Returns the numeric value of the cell
specified by the arguments.  The \texttt{string} argument specifies
the column (\texttt{A}, \texttt{B}, etc.) and \texttt{n} specifies the
row number.  If either of the arguments are outside of the worksheet
limits or the cell has no numeric value, the function returns 0.
 
\fnstart{@pi} Returns the value of $\pi$
(3.141592654...).

\fnstart{@pmt(pv,i,term)} Returns the payment for an ordinary annuity
with the payment made at the end of each term.  The arguments are the
principal present value of the loan amount, \texttt{pv}, the periodic
interest rate, \texttt{i}, and the \texttt{term} (number of periods)
for paying off the loan.  The period for expressing the interest rate
and the term must be the same.  E.g., if \texttt{term} in is months,
\texttt{i} must be given on a monthly basis.

\fnstart{@pow(x,y)} Returns the result of $x$ raised to the power $y$,
i.e.\  $x^y$.  $x$ must be nonnegative.

\fnstart{@prod(crange)} Returns the product of all the nonblank cells
in the given (conditional) range.

\fnstart{@pv(pmt,i,term)} Returns the present value of
an ordinary annuity with the payment made at the end of each period, at
a fixed interest rate.  The arguments are the periodic payment amount,
the interest rate, and the term (number of payments).  The period for
expressing the interest rate and \texttt{term} must be the same.

\fnstart{@rand} (No argument and no parentheses required.)  Returns a
pseudo-random number from the uniform distribution on the range 0 to
100.

\fnstart{@rnd(x)} Returns the value that rounds off $x$ to the nearest
integer.

\fnstart{@rtd(x)} Converts the angle measurement $x$ in radians to
degrees.

\fnstart{@second(date)} Returns the number of seconds since the last
full minute from a unix date.

\fnstart{@sin(x)} Returns the sine of $x$, which is an angle in
radians.

\fnstart{@sqrt(x)} Returns the square root of $x$, for $x \ge 0$.

\fnstart{@stddev(crange)} Returns the sample standard deviation of the
cell values in \texttt{crange}.

\fnstart{@stindex(n,range)} Returns the string contents of a cell
specified by the index number \texttt{n} and \texttt{range}, which is
any single row or column in the worksheet.  The range cells are
numbered from 1 to $n$, starting with the leftmost cell in the row or
the topmost cell in the column.

\fnstart{@ston(string)} Converts \texttt{string} (which must be a
valid string representation of a number) to its numeric
value. 

\fnstart{@sval(string,n)} Returns the string value of the cell
specified by the arguments.  The \texttt{string} argument specifies
the column (\texttt{A}, \texttt{B}, etc.) and the argument \texttt{n}
specifies the row number.  If either of the arguments are outside of
the worksheet limits or the cell has no string value, the function
returns a null string.

\fnstart{@substr(string,position1,position2)} Returns the characters
from \texttt{position1} through and including \texttt{position2} from
the designated \texttt{string}.  The first character in
\texttt{string} is at position number 1.  If \texttt{position2} is
greater than the length of the string, \texttt{position2} is the
length of the string.  If \texttt{position1} is less than 1 or greater
than \texttt{position2}, the function returns the null string.

\fnstart{@sum(crange)} Returns the sum of all the nonblank cells in the
given \texttt{crange}.  The function ignores empty cells and treats
labels as 0.

\fnstart{@tan(x)} Returns the tangent of $x$, an angle in radians.

\fnstart{@year(date)} Returns the year from the unix date.  Valid
years start with 1970; the latest valid year is system dependent.

\subsection{Conditional Ranges}
\label{condrange}

An ordinary range of cells is something like \texttt{B5:C20}; a
\textit{conditional range} is a range plus a logical expression,
separated from the range itself by a hash mark, for instance
\verb+B5:C20#A5=1+.  A conditional range says, in effect, ``Operate
on each cell in the given range if and only if the associated
expression evaluates as true (1) for that cell.''  The logical
expression may be compound (see page~\pageref{logical_ops}).

This feature offers considerable flexibility: it allows you to treat
the spreadsheet as a database, selecting cases for processing based on
some criterion of interest.  The functions that support the
conditional range syntax are \texttt{sum}, \texttt{product},
\texttt{count}, \texttt{max}, \texttt{min}, \texttt{avg} and
\texttt{stddev}.  We illustrate the idea with reference to the
\texttt{sum} function.

Suppose we have an expenses worksheet, with a column of disbursements
and a column of ``budget codes'' indicating the type of expense.  For
example a code of 1 might indicate spending on food and 2 indicate
entertainment.  Now we'd like to find the total expenditure in each
category.  Let's say the budget codes are in column \texttt{A},
starting at row 0 and running through row 20, and the expense items
are next door in column \texttt{B}.  The formula for the sum of food
expenses is \verb+@sum(B0:B20#A0=1)+, and the sum of entertainment
expenses is \verb+@sum(B0:B20#A0=2)+.  Note that the logical condition
is ``advanced'' appropriately as evaluation proceeds down the column:
\texttt{A0} is the first cell tested, but at the next step \texttt{A1}
is tested, and so on.

\end{document}
